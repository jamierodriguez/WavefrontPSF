% -----------------------------*- LaTeX -*------------------------------
\documentclass[aas_macros,preprint]{aastex}%preprint2]{aastex}
% ------------------------------------------------------------------------
% Packages
% ------------------------------------------------------------------------
\usepackage{amsmath,amssymb,graphicx,nicefrac,mathtools}
%\usepackage{scribe}
%\usepackage{amsfonts}
\usepackage{apjfonts}
%\usepackage{mathpazo}
\usepackage{geometry}%[body={7in, 9in},left=1in,right=1in]{geometry}
\usepackage{enumitem}

%~~~~~~~~~~~~~~~
% Things I have added
%~~~~~~~~~~~~~~~
\usepackage{hyperref}

\usepackage{biblatex}
% http://tex.stackexchange.com/questions/5091/what-to-do-to-switch-to-biblatex
\addbibresource{/Users/cpd/Dropbox/Papers/database.bib}

\usepackage[textsize=\Small]{todonotes}
\setlength{\marginparwidth}{2cm} % fix in todonotes doc for oddly placed todos
\reversemarginpar % put todonotes on left side of page

% ------------------------------------------------------------------------
% Macros
% ------------------------------------------------------------------------
%~~~~~~~~~~~~~~~
% Environment shortcuts
%~~~~~~~~~~~~~~~
\def\balign#1\ealign{\begin{align}#1\end{align}}
\def\baligns#1\ealigns{\begin{align*}#1\end{align*}}
\def\bitemize#1\eitemize{\begin{itemize}#1\end{itemize}}
\def\benumerate#1\eenumerate{\begin{enumerate}#1\end{enumerate}}
%~~~~~~~~~~~~~~~
% Text with quads around it
%~~~~~~~~~~~~~~~
\newcommand{\qtext}[1]{\quad\text{#1}\quad}
%~~~~~~~~~~~~~~~
% Shorthand for math formatting
%~~~~~~~~~~~~~~~
\def\mbb#1{\mathbb{#1}}
\def\mbi#1{\boldsymbol{#1}} % Bold and italic (math bold italic)
\def\mbf#1{\mathbf{#1}}
\def\mc#1{\mathcal{#1}}
\def\mrm#1{\mathrm{#1}}
\def\tbf#1{\textbf{#1}}
\def\tsc#1{\textsc{#1}}
%~~~~~~~~~~~~~~~
% Common sets
%~~~~~~~~~~~~~~~
\def\reals{\mathbb{R}} % Real number symbol
\def\integers{\mathbb{Z}} % Integer symbol
\def\rationals{\mathbb{Q}} % Rational numbers
\def\naturals{\mathbb{N}} % Natural numbers
\def\complex{\mathbb{C}} % Complex numbers
%~~~~~~~~~~~~~~~
% Common functions
%~~~~~~~~~~~~~~~
\renewcommand{\exp}[1]{\operatorname{exp}\left(#1\right)} % Exponential
\def\indic#1{\mbb{I}\left({#1}\right)} % Indicator function
\providecommand{\argmax}{\mathop\mathrm{arg max}} % Defining math symbols
\providecommand{\argmin}{\mathop\mathrm{arg min}}
\providecommand{\arccos}{\mathop\mathrm{arccos}}
\providecommand{\dom}{\mathop\mathrm{dom}} % Domain
\providecommand{\range}{\mathop\mathrm{range}} % Range
\providecommand{\diag}{\mathop\mathrm{diag}}
\providecommand{\tr}{\mathop\mathrm{tr}}
\providecommand{\abs}{\mathop\mathrm{abs}}
\providecommand{\card}{\mathop\mathrm{card}}
\providecommand{\sign}{\mathop\mathrm{sign}}
\def\rank#1{\mathrm{rank}({#1})}
\def\supp#1{\mathrm{supp}({#1})}
%~~~~~~~~~~~~~~~
% Common probability symbols
%~~~~~~~~~~~~~~~
\def\E{\mathbb{E}} % Expectation symbol
\def\Earg#1{\E\left[{#1}\right]}
\def\Esubarg#1#2{\E_{#1}\left[{#2}\right]}
\def\P{\mathbb{P}} % Probability symbol
\def\Parg#1{\P\left({#1}\right)}
\def\Psubarg#1#2{\P_{#1}\left[{#2}\right]}
\def\Cov{\mrm{Cov}} % Covariance symbol
\def\Covarg#1{\Cov\left[{#1}\right]}
\def\Covsubarg#1#2{\Cov_{#1}\left[{#2}\right]}
\newcommand{\family}{\mathcal{P}} % probability family / statistical model
\newcommand{\iid}{\stackrel{\mathrm{iid}}{\sim}}
\newcommand{\ind}{\stackrel{\mathrm{ind}}{\sim}}
\def\E{\mathbb{E}} % Expectation symbol
\def\Earg#1{\E\left[{#1}\right]}
\def\Esubarg#1#2{\E_{#1}\left[{#2}\right]}
\def\P{\mathbb{P}} % Probability symbol
\def\Parg#1{\P\left({#1}\right)}
\def\Psubarg#1#2{\P_{#1}\left[{#2}\right]}
\def\Cov{\mrm{Cov}} % Covariance symbol
\def\Covarg#1{\Cov\left[{#1}\right]}
\def\Covsubarg#1#2{\Cov_{#1}\left[{#2}\right]}
\newcommand{\model}{\mathcal{P}} % probability family / statistical model
%~~~~~~~~~~~~~~~
% Distributions
%~~~~~~~~~~~~~~~
\def\Gsn{\mathcal{N}}
\def\Ber{\textnormal{Ber}}
\def\Bin{\textnormal{Bin}}
\def\Unif{\textnormal{Unif}}
\def\Mult{\textnormal{Mult}}
\def\NegMult{\textnormal{NegMult}}
\def\Dir{\textnormal{Dir}}
\def\Bet{\textnormal{Beta}}
\def\Gam{\textnormal{Gamma}}
\def\Poi{\textnormal{Poi}}
\def\HypGeo{\textnormal{HypGeo}}
\def\GEM{\textnormal{GEM}}
\def\BP{\textnormal{BP}}
\def\DP{\textnormal{DP}}
\def\BeP{\textnormal{BeP}}
\def\Exp{\textnormal{Exp}}
%~~~~~~~~~~~~~~~
% Theorem-like environments
%~~~~~~~~~~~~~~~
%\theoremstyle{definition}
\newtheorem{definition}{Definition}
\newtheorem{example}{Example}
\newtheorem{problem}{Problem}

%~~~~~~~~~~~~~~~
% My Macros
%~~~~~~~~~~~~~~~

% New definition of square root: it renames \sqrt as \oldsqrt
\let\oldsqrt\sqrt
% it defines the new \sqrt in terms of the old one
\def\sqrt{\mathpalette\DHLhksqrt} \def\DHLhksqrt#1#2{%
  \setbox0=\hbox{$#1\oldsqrt{#2\,}$}\dimen0=\ht0
  \advance\dimen0-0.2\ht0 \setbox2=\hbox{\vrule height\ht0 depth
    -\dimen0}%
  {\box0\lower0.4pt\box2}}
\newcommand{\figfig}[3]{
    \begin{figure}
      \begin{center}
        \includegraphics[width=0.9\textwidth,height=0.35\textheight,keepaspectratio]
        {Figures/{#1}}
      \end{center}
      \caption{{#2}}
      \label{#3}
    \end{figure}}

\newcommand{\dd}{\mathrm{d}}
\newcommand{\arcsec}{\mathrm{arcsec}} \newcommand
{\fig}[1]{Figure~\ref{#1}} \newcommand {\sect}[1]{Section~\ref{#1}}
\newcommand {\eq}[1]{Equation~\eqref{#1}}
\newcommand{\tab}[1]{Table~\ref{#1}}

\newcommand*{\DOT}{.}

\newcommand{\average}[1]{\ensuremath{\langle {#1} \rangle}}

% ----------------------------------------------------------------------
% Header information
% ------------------------------------------------------------------------
\begin{document}
\title{ Zernike Conventions }
\author{ Christopher Davis }
\affil{}
\email{}
\date{ \today }

\maketitle

\begin{abstract}
  Abstract
\end{abstract}
\keywords{Zernike Polynomials}

\tableofcontents

% scribe package header stuff
% \course{}
% \coursetitle{}
% \semester{}
% \lecturer{}
% \scribe{}         % your name
% \lecturenumber{}               % lecture number
% \lecturedate{}       % month and day of lecture (omit year)

% todo stuff
\todototoc
\listoftodos



% ----------------------------------------------------------------------
% Body of the document
% ------------------------------------------------------------------------

\section{Wavefront Model}

  The basic model for light propagation is Fresnel Diffraction:
  \begin{equation}
    \Psi(x, y, z, t) = \exp{\frac{\imath 2 \pi}{\lambda} (z - t)}
    u_z(x, y) \ ,
  \end{equation}
  where we take the initial wavefront $u_{z=0}$ to be
  \begin{equation}
    u_0(x,y) \propto \exp{P(x, y)} \exp{\imath \Phi(x, y))} \ ,
  \end{equation}
  where $r$ is the two-dimensional radial coordinate, $P$ is the pupil
  function, and $\Phi$ are the aberrations. $u_0$ is convolved with an
  atmospheric seeing kernel (Kolmogorov), then propagated to the focal plane.
  We measure $| u_z(x,y)|^2$. Aberrations in the wavefront ($\Phi \neq 0$)
  can introduce spurious shapes.


  \subsection{alt:}


The basic model for light propagation is Fresnel Diffraction:
\begin{equation}
  \Psi(x, y, z, t) = \exp{\frac{\imath 2 \pi}{\lambda} (z - t)}
  u_z(x, y) \ ,
\end{equation}
where we take the initial wavefront $u_{z=0}$ to be
\begin{equation}
  u_0(x,y) \propto \exp{P(x, y)} \exp{\imath \Phi(x, y))} \ ,
\end{equation}
where $r$ is the two-dimensional radial coordinate, $P$ is the pupil
function, and $\Phi$ are the aberrations.

$u_0$ is convolved with an atmospheric seeing kernel (Kolmogorov), then
propagated to the focal plane:
\begin{equation}
  \label{eq:u}
  u_z = \frac{\exp{\imath \pi r^2 / \lambda z}}{\imath \lambda z} \mathrm{FT}\left[
  \exp{\frac{\imath \pi r^2}{\lambda z}} u_0 (x, y) \right] \ .
\end{equation}
What we measure is the irradiance:
\begin{equation}
  I_z(x, y) = | u_z (x, y) |^2 .
\end{equation}
Aberrations in the wavefront ($\Phi \neq 0$) can introduce spurious shapes.


  These aberrations vary across the focal plane; $d, a \dots$
  are really $d(x, y)$, functions of focal plane location. Call $d_0, a_0
  \dots$ the `normal' optics contribution to wavefront aberrations. Defocus and
  misalignment of the focal plane can be characterized by linear shifts and
  tilts of the normal wavefront, so that for some exposure $i$ the defocus $d$
  is:
  \begin{equation}
    \label{eq:deviation}
    d_i(x, y) = (1 + \Delta_d + \theta_{y,d} x + \theta_{x,d} y)d_0(x, y) \ .
  \end{equation}
  Measurements of these variations using the focus chips is how the active
  optics system works.

  We would like to obtain this same information for the focal plane, not just
  the alignment chips. But because the focal plane is near focus, directly
  modeling $\Phi$ is difficult.  Instead, parameterize the shape into a sum of
  moments:
  \begin{equation}
    \label{eq:moments}
    M_{pq, z} = \frac{\int \dd A \ w(x,y) I_z(x,y) (x-\bar{x})^p
    (y-\bar{y})^q}{\int \dd A \ w(x,y) I_z(x,y)} \ ,
  \end{equation}
  where $\bar{x}$ and $\bar{y}$ are the centroids of the image and $w$ is a
  weighting function.

  We measure the moments here using the \textit{hsm} weighting scheme, an
  iterative weighting optimized for measuring second moments used in galsim.

\section{Zernikes -- Noll, Complex Number}

Want table like one from cpd2014MayCollaboration but also including azimuthal
etc degree and form as a complex number $z$ and $x + \imath y$

also astigmatism -x and y as well as trefoil need to be renamed

\begin{center}
\begin{tabular}{ l l l r l }
  Type & Noll & $(n,\ m)$ & Variable & Polar Polynomial \\
  \hline \hline
  Piston            & $1$  & $(0,\ 0)$ & $0$      & $1$                                 \\
  Tilt-$0$          & $2$  & $(0,\ 1)$ & $0$      & $2 r \cos \theta$                   \\
  Tilt-$90$          & $3$  & $(0,\ 1)$ & $0$      & $2 r \sin \theta$                   \\
  Defocus           & $4$  & $(2,\ 0)$ & $d$      & $\sqrt{3} (2 r^2 - 1)$              \\
  Astigmatism-$45$   & $5$  & $(2,\ 2)$ & $\Im[a]$ & $\sqrt{6} r^2 \sin 2 \theta$        \\
  Astigmatism-$0$   & $6$  & $(2,\ 2)$ & $\Re[a]$ & $\sqrt{6} r^2 \cos 2 \theta$        \\
  Coma-$90$          & $7$  & $(3,\ 1)$ & $\Im[c]$ & $\sqrt{8} (3 r^3 - 2r) \sin \theta$ \\
  Coma-$0$          & $8$  & $(3,\ 1)$ & $\Re[c]$ & $\sqrt{8} (3 r^3 - 2r) \cos \theta$ \\
  Trefoil-$30$       & $9$  & $(3,\ 3)$ & $\Im[t]$ & $\sqrt{8} r^3 \sin 3 \theta$        \\
  Trefoil-$0$       & $10$ & $(3,\ 3)$ & $\Re[t]$ & $\sqrt{8} r^3 \cos 3 \theta$        \\
  Spherical Defocus & $11$ & $(4,\ 0)$ & $s$      & $\sqrt{5} (6 r^4 - 6 r^2 + 1)$      \\
\end{tabular}
\end{center}

\section{Moments and Adaptive Moments}

both the conventional definition and talk about how I use Hirata and Seljak
adaptive moments.

\section{Whisker Conventions}
Lift from whiskerconvention for $w$?

\section{Higher Order Moments}

  In analogy with
  gravitational shear and flexion, these moments can be decomposed into
  linear combinations with convenient rotational symmetries:
  \begin{align}
    e_{0} &= M_{20} + M_{02} \\
    e_{1} &= M_{20} - M_{02} \\
    e_{2} &= 2 M_{11} \\
    \zeta_{1} &= M_{30} + M_{12} \\
    \zeta_{2} &= M_{03} + M_{21} \\
   \delta_{1} &= M_{30} - 3 M_{12} \\
   \delta_{2} &= -M_{03} + 3 M_{21} \\
  \end{align}
  $\boldsymbol{\epsilon} = (e_1 + \imath e_2) / e_0$ is one of the common definitions for
  ellipticity, while $\zeta$ and $\delta$ correspond to unnormalized F-1 and
  F-3 flexion, respectively.\todo{Is it $\epsilon$ or $\chi$?}


  Wavefront aberrations lead to measurable moments.
  Plugging \eqref{eq:u} into \eqref{eq:moments} and carrying out the math for
  the Zernike aberrations defocus, astigmatism, coma, trefoil, and spherical
  defocus ($d, a, c, t, s$) we find analytically:
  \begin{align*}
    e_0 &=&&24 d^2 + 16 \sqrt{15} ds + 120 s^2 + 12 |a|^2 + 56 |c|^2 + 24 |t|^2 \\
    \bold{e} = e_1 + \imath e_2 &=&&8 \sqrt{2} (3 d + \sqrt{15} s) a + 32 c^2 +
      45 \bar{c} t \\
      \boldsymbol{\zeta} = \zeta_1 + \imath \zeta_2 &=&&
      64(3 d + 2 \sqrt{15}s) (2 a \bar{c} + \bar{a} t)+ \\
    &&&16 \sqrt{2} (18 d^2 + 22 \sqrt{15} ds + 120 s^2 + 6 |a|^2 + 25 |c|^2 + 15 |t|^2) c \\
    &&&+ 48 \sqrt{2} a^2 \bar{t} + 240 \sqrt{2} \bar{c}^2 t \\
      \boldsymbol{\delta} = \delta_1 + \imath \delta_2 &=&&192 (3d + 2 \sqrt{15}
      s) a c \\
      &&&+144 \sqrt{2} (2 d^2 + 2\sqrt{15} ds + 12 s^2 + 5 |c|^2)t \\
      &&&+144 \sqrt{2} a^2
      \bar{c} + 160 \sqrt{2} c^3 \\
  \end{align*}

\section{Zernike v Seidel}

maybe work out the field aberration stuff with seidel first and then with
zernike (the maths certainly will be at least a /little/ simpler...)

\end{document}


